\usepackage[utf8]{inputenc}
\usepackage{tabularray}	% for "tblr"
\usepackage{graphicx}

% Define the colors used by DNP for branding
\usepackage{xcolor}
\definecolor{dnpblue}{RGB}{0,26,174}
\definecolor{dnplightblue}{RGB}{221,226,255}
\usecolortheme[named=dnpblue]{structure}

% No navigation symbols ("next slide"/"previous slide")
\setbeamertemplate{navigation symbols}{}

% make the footline DNP blue
\setbeamercolor{palette primary}{bg=dnpblue,fg=white}
\setbeamercolor{palette secondary}{bg=dnpblue,fg=white}
\setbeamercolor{palette tertiary}{bg=dnpblue,fg=white}

% define associative array functionality, e.g.
%   \storedata\mydata{{one}{two}{three}}
%   \getdata[2]\mydata   % returns two
\newcount\tmpnum
\def\storedata#1#2{\tmpnum=0 \edef\tmp{\string#1}\storedataA#2\end}
\def\storedataA#1{\advance\tmpnum by1
	\ifx\end#1\else
	\expandafter\def\csname data:\tmp:\the\tmpnum\endcsname{#1}%
	\expandafter\storedataA\fi
}
\def\getdata[#1]#2{\csname data:\string#2:#1\endcsname}

% define the fading used on the title slide
\usepackage{tikz}
\usetikzlibrary{fadings,calc}
\tikzfading[name=dnpfading,top color=transparent!100,bottom color=transparent!0]

\tikzset{fotopunkt/.style={scale=.7,inner sep=1pt,text=white,draw=white,fill=dnpblue,circle}}

% draw the zoom triangle between a point of interest and its photo
%  * number of the point (#1)
%  * photo (as a TikZ node) (#2)
\newcommand\zoomtriangle[2]{
	\tikz[remember picture,overlay]{
		\coordinate (top) at ($(map.north west)!\getdata[#1]\xcoords!(map.north east)$);
		\coordinate (bottom) at ($(map.south west)!\getdata[#1]\xcoords!(map.south east)$);
		\coordinate (target) at ($(top)!\getdata[#1]\ycoords!(bottom)$);
		\fill[gray,opacity=.35] (target) -- (#2.north west) -- (#2.south west) -- cycle;
		\node[fotopunkt] at (target) {#1};
	}
}


% bulletlist is an itemized list which has a blue bullet before each item
\setbeamertemplate{itemize items}{$\bullet$}
\newenvironment{bulletlist}[1]{\begin{itemize}\setlength{\itemsep}{#1}}{\end{itemize}}

% define a new environment "mapframe" which consists of
%  * a title (#1)
%  * a left column featuring an image (#2) of width #3
%  * a right column containing the content of the environment
% example:
% \begin{mapframe}{Frame title}{/path/to/map.png}{.5\textwidth}
	%     (content of the right column)
	% \end{mapframe}
\usepackage{environ,calc}
\NewEnviron{mapframe}[3]{\begin{frame}{#1}
		\begin{minipage}{#3}
			\tikz[remember picture]{\node[inner sep=0] (map) at (0, 0) {\includegraphics[width=\linewidth]{#2}};}
		\end{minipage}\hfill%
		\begin{minipage}{\dimexpr0.99\linewidth-#3}\BODY\end{minipage}
\end{frame}}

% logo in the bottom right corner
\logo{\includegraphics[width=.1\textwidth]{../Bilder/Logo.png}}
\newcommand{\nologo}{\setbeamertemplate{logo}{}} % command to set the logo to nothing

\setbeamertemplate{date}{}

\setbeamertemplate{title page}{
	\begin{tikzpicture}[overlay, remember picture, shift={(current page.south west)}]
		\node[inner sep=0pt] at (10, \paperheight/2) {\includegraphics[height=\paperheight]{../Karten/titelbild.pdf}};
		\fill[dnplightblue,path fading=dnpfading] (0, 0) rectangle (\paperwidth, \paperheight);
		\fill[dnpblue] (0, 0) rectangle  (5.3, \paperheight);
		\node[white,anchor=west,text width=4.9cm] at (.2, 7) {\inserttitle{}};
		\node[white,anchor=west,text width=4.9cm] at (.2, 5) {\scriptsize \insertsubtitle{}};
		\node[white,anchor=south west] at (.2, .2) {\tiny \insertdate{}};
		\node[anchor=south east] at (\paperwidth, .1) {\insertlogo{}};
	\end{tikzpicture}
	\addtocounter{framenumber}{-1} % exclude titlepage from page numbering
}
